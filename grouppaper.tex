\documentclass[paper=a4, fontsize=11pt]{scrartcl} % A4 paper and 11pt font size

\usepackage[document]{ragged2e}
\usepackage[T1]{fontenc} % Use 8-bit encoding that has 256 glyphs
\usepackage{fourier} % Use the Adobe Utopia font for the document - comment this line to return to the LaTeX default
\usepackage[english]{babel} % English language/hyphenation
\usepackage{amsmath,amsfonts,amsthm} % Math packages

\usepackage{lipsum} % Used for inserting dummy 'Lorem ipsum' text into the template

\usepackage{sectsty} % Allows customizing section commands
\allsectionsfont{\centering \normalfont\scshape} % Make all sections centered, the default font and small caps

\topmargin 0pt
\advance \topmargin by -\headheight
\advance \topmargin by -\headsep
\textheight 8.9in
\oddsidemargin 0pt
\evensidemargin \oddsidemargin
\marginparwidth 0.5in
\textwidth 6.5in

\usepackage{fancyhdr} % Custom headers and footers

\pagestyle{fancyplain} % Makes all pages in the document conform to the custom headers and footers
\fancyhead{} % No page header - if you want one, create it in the same way as the footers below
\fancyfoot[L]{} % Empty left footer
\fancyfoot[C]{} % Empty center footer
\fancyfoot[R]{\thepage} % Page numbering for right footer
\renewcommand{\headrulewidth}{0pt} % Remove header underlines
\renewcommand{\footrulewidth}{0pt} % Remove footer underlines
\setlength{\headheight}{13.6pt} % Customize the height of the header

\numberwithin{equation}{section} % Number equations within sections (i.e. 1.1, 1.2, 2.1, 2.2 instead of 1, 2, 3, 4)
\numberwithin{figure}{section} % Number figures within sections (i.e. 1.1, 1.2, 2.1, 2.2 instead of 1, 2, 3, 4)
\numberwithin{table}{section} % Number tables within sections (i.e. 1.1, 1.2, 2.1, 2.2 instead of 1, 2, 3, 4)

\setlength\parindent{0pt} % Removes all indentation from paragraphs - comment this line for an assignment with lots of text

%----------------------------------------------------------------------------------------
%	TITLE SECTION
%----------------------------------------------------------------------------------------

\newcommand{\horrule}[1]{\rule{\linewidth}{#1}} % Create horizontal rule command with 1 argument of height

\title{	
\normalfont \normalsize 
\textsc{project for CS388P: Parallel Algorithms (Fall 2016)} 
\horrule{0.5pt} \\[0.4cm] % Thin top horizontal rule
\huge We Want More Time and Less Work \\ % Write your project title here
\horrule{2pt} \\[0.5cm] % Thick bottom horizontal rule
}

\author{Matthew Hudson \hspace{5mm} Walter Xia }	%Write the name of all your team members here

\date{\normalsize\today} % Today's date or a custom date

\begin{document}

\maketitle % Print the title

%----------------------------------------------------------------------------------------
%	ABSTRACT
%----------------------------------------------------------------------------------------
\justify
\textbf{Abstract}

\begin{abstract}
Our paper is \textit{Time-Work Tradeoffs for Parallel Algorithms}, \textit{Spencer\cite{S97}}. The high-level idea that serves as the theme of \textit{Spencer\cite{S97}} is the observation that there exists algorithms with the property that the amount of work they perform is inversely related to the amount of time they take. In particular, \textit{Spencer\cite{S97}} discusses the following problems:

\begin{itemize}
\item Solving Triangular Systems of Linear Equations
\item Topological Sort
\item Breadth-First Search in Directed Graphs
\item Strongly Connected Components
\item Single Source Shortest Path
\end{itemize}

In particular, it presents a few new list data structures called \textit{nearby lists}, \textit{frontier lists}, \textit{inverse nearby lists}, and \textit{inverse frontier lists} that provide the necessary bookkeeping for the subsequent algorithms to be presented. \\
\textbf{\textit{/* Note to Matt: I may have omitted stuff from your parts. For example, stuff related to probability. If you think we should add anything here, please do so. */}}

\end{abstract}

%----------------------------------------------------------------------------------------
%	INTRODUCTION
%----------------------------------------------------------------------------------------

\section{Introduction}

\textit{Spencer\cite{S97}} focuses on the problems for which there is no known efficient parallel algorithm that solves them. That is, there is no known parallel algorithm that is able to solve them in work that is within a logarithmic factor of the fastest sequential algorithm, where work is the product of the running time and the processor count. \textit{Spencer\cite{S97}} develops parallel algorithms that solves the following problems presented in Karp-Ramachandran\cite{KR90}:

\begin{itemize}
\item Directed Spanning Tree
\item Breadth-First Search
\item Topological Sort
\item Cycle Detection for Directed Graphs
\item Strong Connected Components
\item Single Source Shortest Paths with Non-Negative Edges
\end{itemize}

His algorithms exhibit a time-work trade-off, that is, the longer they run, the less work they perform. This phenomenon exists due of the fact that certain speculative computations can be eliminated as time progresses, thus reducing work. One algorithm solves topological sort and the other algorithm solves breadth first search. It was mentioned in passing that the breadth first search algorithm can solve directed spanning trees and the topological sort algorithm can solve cycle detection for directed graphs. Furthermore, the breadth first search algorithm can be applied to strongly connected components and extended to single source shortest paths with non-negative edges.\\

\textbf{\textit{/* Note: We may or may not need these definitions here. */}}\\
We next reproduce three definitions given in \textit{Spencer\cite{S97}}:\\
\indent\emph{Definition.} A vertex $v$ \textit{reaches} a vertex $u$ iff there is a path from $v$ to $u$. Conversely, $u$ is \textit{reachable} from $v$ iff there is path from $v$ to $u$.\\
\indent\emph{Definition.} Let $G = (V, E)$ be a graph and let $R \subseteq V$. $G[R] = (R, E')$, where $E' \subseteq R \times R$.\\
\indent\emph{Definition.} Two vertices $u,v$ are \textit{identified} by replacing them with a single vertex $uv$. The outgoing edges of both $u$ and $v$ becomes the outgoing edges of $uv$ and the incoming edges of both $u$ and $v$ become the incoming edges of $uv$.

%----------------------------------------------------------------------------------------
%	RESULTS
%----------------------------------------------------------------------------------------

\section{Results and Techniques of Assigned paper}

Highlight the major contributions and results of the assigned paper.

\subsection{Techniques}
Give an overview of the major techniques that were used in the paper to achieve these results. 


%----------------------------------------------------------------------------------------
%	LITERATURE SURVEY
%----------------------------------------------------------------------------------------

\section{Significant Related Results}

Cite the three most significant related results. Here are some sample citations.
Example: this paper \cite{S97}.

Highlight the major contributions and results of these papers, and their relation to your assigned
paper. Indicate which, if any, of these paperswere cited in your assigned paper.
Also, you must cite the journal version of a conference paper, if the journal version has appeared.

Please note that while Google Scholar and other web tools are very useful in finding related results, they may not always give a pointer to the journal version or even the conference version. You need to locate and cite the latest version. 



%----------------------------------------------------------------------------------------
%	PRESENTATION TOPICS
%----------------------------------------------------------------------------------------

\section{Presentation Topics}

Here each team member should cite the paper from which they will present technical results in their
presentation, and should indicate the exact theorem(s)/lemma(s) whose proof(s) they will present.

%----------------------------------------------------------------------------------------
%	FURTHER RESEARCH
%----------------------------------------------------------------------------------------

\section{Further Research Directions}
Give an overview of the current open problems and future research directions related to the work in the assigned paper.



%\clearpage
%----------------------------------------------------------------------------------------
%	BIBLIOGRAPHY
%----------------------------------------------------------------------------------------

\bibliographystyle{abbrv}
\bibliography{spencer,karp_ramachandran}

\end{document}